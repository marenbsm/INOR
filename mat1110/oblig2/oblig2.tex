\documentclass[11pt]{article}
\usepackage[utf8]{inputenc}
\usepackage{listings}
\usepackage{amsmath}
\usepackage{tikz}
\usepackage{amssymb}
\usepackage{multicol}
\renewcommand{\vec}[1]{\mathbf{#1}}
\newcommand{\bpm}{\begin{pmatrix}}
\newcommand{\epm}{\end{pmatrix}}
\title{MAT1110 Oblig2}
\author{Erik Øystein Gåserud}
\date{\today}

\begin{document}

	\maketitle
	\section*{Oppgave 1}
	\subsection*{a)}
	Av oppgaven har vi fått opplyst:
	\begin{multicols}{3} \noindent
		\begin{align*}
		A = \bpm 4 & 6 \\ 6 & -1 \epm
		\end{align*}
		\begin{align*}
			\vec{v}_{1} = \bpm 3 \\ 2 \epm
		\end{align*}
		\begin{align*}
			\vec{v}_{2} = \bpm 2 \\ -3 \epm
		\end{align*}
	\end{multicols}
	Vi ganger ut $A\vec{v}_{1}$ og $A\vec{v}_{2}$:
	\begin{multicols}{2} \noindent
		\begin{align*}
			A\vec{v}_{1} = \bpm 24 \\ 16 \epm = A\lambda_{1}
		\end{align*}
		\begin{align*}
			A\vec{v}_{2} = \bpm -10 \\ 15 \epm = A\lambda_{2}\\
		\end{align*}
	\end{multicols}
		som gir ligningene for egenverdiene $\lambda_{1}$ og $\lambda_{2}$:
	\begin{multicols}{2} \noindent
		\begin{align*}
			3\lambda_{1} = 24&\wedge2\lambda_{1} = 16 \\
			&\Downarrow \\
			\lambda_{1} &= 8 \\
		\end{align*}
		\begin{align*}
			3\lambda_{2} = -10&\wedge2\lambda_{2} = 15 \\
			&\Downarrow \\
			\lambda_{2} &= -5\\
		\end{align*}
	\end{multicols}
%	\newpage
	\subsection*{b)}
	Dersom $\vec{v}$ er en egenvektor for en $n\,x\,n$ matrise $A$ med en egenverdi $\lambda$, så er også enhver parallell vektor, $c\vec{v} \, der \, c\not=0$, en egenvektor med egenverdi $\lambda$ siden :
	\begin{align*}
	A(c\vec{v}) = c(A\vec{v}) = c(\lambda\vec{v}) = \lambda(c\vec{v})
	\end{align*}


\end{document}